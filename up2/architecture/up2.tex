\documentclass[a4paper]{article}

\usepackage{fullpage} % Package to use full page
\usepackage{parskip} % Package to tweak paragraph skipping
\usepackage{tikz} % Package for drawing
\usepackage{amsmath}
\usepackage{hyperref}

\title{up2}
\author{Ashley J. Robinson}
\date{\today}

\begin{document}

\maketitle

\section{}

% No need to have a JF and JB if ADDI ? = +1 + Imm can force the flag
% Link register can exist with PC and can be shift register loaded. ULB << << << ULS
% Relative jumps link to static jumps so a branch can be static
% Enought room to hold static jump over the table in 0x04..0x07
% Could the LR just be the IR? So intrustions shifted in can be as wide as the PC and address width

\begin{table}[]
    \centering
    \caption{My caption}
    \label{my-label}
    \begin{tabular}{|l|l|l|l|l|l|l|}
        \hline
        \textbf{Opcode}     &   \textbf{Opcode} &                       &                       &                       &                                       &           \\ \hline
        \textbf{Hex}        &   \textbf{Binary} &   \textbf{Mnemonic}   &   \textbf{Trailers}   &    &   \textbf{Description}                &           \\ \hline
        0x0                 &   0000            &   ADDI                &   2                   &               &   <R0,R1,R2,?> = <+1,R0> + Imm        &           \\ \hline
        0x1                 &   0001            &   ADD                 &   1                   &               &   <R0,R1,R2,?> = <R1,R2> + <+1,R0>    &           \\ \hline
        0x2                 &   0010            &   SUB                 &   1                   &               &   <R0,R1,R2,?> = <R1,R2> - <+1,R0>           &           \\ \hline
        0x3                 &   0011            &   JB                &   1                   &               &   PC = PC - Imm           &           \\ \hline
        0x4                 &   0100            &   NAND                &  1             &           &           &           \\ \hline
        0x5                 &   0101            &   LOAD                &   1                   &              &   DATA[<R0,R1,R2>] = <R0,R1,R2>           &           \\ \hline
        0x6                 &   0110            &   STORE               &  1               &                       &   <R0,R1,R2> = DATA[<R0,R1,R2>]           &           \\ \hline
        0x7                 &   0111            &   LINK                &   0           &           &   LR = PC          &           \\ \hline
        0x8                 &   1000            &   BNE                 &  0            &           &   if(z != 0) PC = PC + 2          &           \\ \hline
        0x9                 &   1001            &   BE                  &  0         &           &  if(z != 1) PC = PC + 2         &           \\ \hline
        0xA                 &   1010            &   JF                  &  1         &           &  PC = PC + Imm         &           \\ \hline
        0xB                 &   1011            &   JB                  &  1         &           &  PC = PC - Imm           &           \\ \hline
        0xC                 &   1100            &   PUSH                &  1         &           &  DATA_STACK[DSP] = <R0,R1,R2,?>;      DSP = DSP - 1;          &           \\ \hline
        0xD                 &   1101            &   POP                 &  1         &           &  <R0,R1,R2,?> = DATA_STACK[DSP+1];    DSP = DSP + 1;           &           \\ \hline
        0xE                 &   1110            &   PCPUSH              &  0         &           &  ADDRESS_STACK[ASP] = PC;             ASP = ASP - 1;           &           \\ \hline
        0xF                 &   1111            &   PCPOP               &  0         &           &  PC = ADDRESS_STACK[ASP+1];           ASP = ASP + 1;            &           \\ \hline
    \end{tabular}
\end{table}
\end{document}
